\documentclass[12pt,a4paper]{article}
\usepackage{ctex}
\usepackage{amsmath,amscd,amsbsy,amssymb,latexsym,url,bm,amsthm}
\usepackage{epsfig,graphicx,subfigure}
\usepackage{enumitem,balance}
\usepackage{wrapfig}
\usepackage{mathrsfs,euscript}
\usepackage[usenames]{xcolor}
\usepackage{hyperref}
\usepackage[vlined,ruled,linesnumbered]{algorithm2e}
\hypersetup{colorlinks=true,linkcolor=black}

\newtheorem{theorem}{Theorem}
\newtheorem{lemma}[theorem]{Lemma}
\newtheorem{proposition}[theorem]{Proposition}
\newtheorem{corollary}[theorem]{Corollary}
\newtheorem{exercise}{Exercise}
\newtheorem*{solution}{Solution}
\newtheorem{definition}{Definition}
\theoremstyle{definition}

\renewcommand{\thefootnote}{\fnsymbol{footnote}}

\newcommand{\postscript}[2]
 {\setlength{\epsfxsize}{#2\hsize}
  \centerline{\epsfbox{#1}}}

\renewcommand{\baselinestretch}{1.0}

\setlength{\oddsidemargin}{-0.365in}
\setlength{\evensidemargin}{-0.365in}
\setlength{\topmargin}{-0.3in}
\setlength{\headheight}{0in}
\setlength{\headsep}{0in}
\setlength{\textheight}{10.1in}
\setlength{\textwidth}{7in}
\makeatletter \renewenvironment{proof}[1][Proof] {\par\pushQED{\qed}\normalfont\topsep6\p@\@plus6\p@\relax\trivlist\item[\hskip\labelsep\bfseries#1\@addpunct{.}]\ignorespaces}{\popQED\endtrivlist\@endpefalse} \makeatother
\makeatletter
\renewenvironment{solution}[1][Solution] {\par\pushQED{\qed}\normalfont\topsep6\p@\@plus6\p@\relax\trivlist\item[\hskip\labelsep\bfseries#1\@addpunct{.}]\ignorespaces}{\popQED\endtrivlist\@endpefalse} \makeatother

\begin{document}
\noindent

%========================================================================
\noindent\framebox[\linewidth]{\shortstack[c]{
\Large{\textbf{Lab06-Graph Exploration}}\vspace{1mm}\\
CS214-Algorithm and Complexity, Xiaofeng Gao, Spring 2019.}}


\begin{center}
\footnotesize{\color{red}$*$ If there is any problem, please contact TA Mingran Peng.}\par
% Please write down your name, student id and email.
\footnotesize{\color{blue}$*$ Name:Lynn Xiao \quad Student ID:\_\_\_\_\_\_\_\_\_ \quad Email: \_\_\_\_\_\_\_\_\_\_\_\_}
\end{center}
\begin{enumerate}
    
    \item
    Given a graph, find the number of Strongly Connected Components in the graph.
    \begin{enumerate}
        \item
         Complete the implementation in the provided C/C++ source code. Notice that in the source code there will be more detailed explanation.{\color{blue}(The source code \emph{SCC.cpp} is attached on the course webpage.)}\par
       \item
        Use the $Gephi$ to draw the graph. If you think the data provided is not beautiful, you can generate your own data. Notice that result of $Gephi$ will be taken into consideration of Best Lab.\par 
        
    \end{enumerate}
%    \begin{proof}
%        Uncomment this block to write your proof.
%    \end{proof}


\item
    Remember the lemma introduced in the course: : $\forall u, v \in V$, intervals $[PRE(u), POST(u)]$, $[PRE(v), POST(v)]$ are either disjoint or one is contained within the other.\par
Prove the lemma.\par
    \begin{proof}
       ~\\
       If $(u,v)\notin$ E, we can easily know that intervals $[PRE(u),POST(u)]$ and $[PRE(v),POST(v)]$ are disjoint.
       
       If $(u,v)\in$ E, assume that we first visit $v$.
       
       If $u$ is not visited before, then $PRE(u)\textgreater PRE(v)$ and $POST(v)\textgreater PRE(u)$ since we call $EXPLORE(G,u)$ in $EXPLORE(G,v)$. So $[PRE(u),POST(u)]$ is contained  within    
       
       $[PRE(v),POST(v)]$.
       
       We can get the same conclusion when $u$ is visited first. So the lemma is proved.
    \end{proof}
\item
Consider there is a network consists $n$ computers. For some pairs of computers, a wire exists in the pair, which means these two computers can communicate with delay $t$.\par
Assume that computer $s$ wants to issue a message to computer $t$, we want to know the minimum time needed to send this message.\par
You need to provide the pseudo code and analyze the time complexity.\par

    \begin{proof}
       ~\\
       We can use BFS to solve this problem, which is shown in Alg.1.
       
        \begin{minipage}[t]{0.9\textwidth}
       	\begin{algorithm}[H]
       		\KwIn{Numbers of computers $n$, computer $s$ and $t$, Graph $G(V,E)$, delay time $T$}
       		\KwOut{The minimum time needed to send message from computer $s$ to $t$ }
       		\BlankLine
       		\caption{Calculate least delay by BFS}
       		\label{Alg-selectionsort}
       		\BlankLine
       		\For{$u\in v$}{$DIST(u)\leftarrow\infty$}
       		
       		$DIST(s)\leftarrow0$
       		
       		$Q\leftarrow[s]$ ($Q$ is a queue)
       		
       		\While{$Q$ is not empty}{
       			$u=EJECT(Q)$
       			
       			\For{$(u,v)\in E$}{
       				\If{$DIST(v)=\infty$}{
       				$INJECT(Q,v)$
       				
       				$DIST(v)\leftarrow DIST(u)+1$
       				\If{$v$=$t$}{\Return $T\times DIST(v)$}
       			}
       			
       		}
       		}
       	\end{algorithm}
       
       Time complexity:
       
       (i) Best case: $O(1)$, if $(s,t)\in E$.
       
       (ii) Worst case: $O(|V|+|E|)$, if every edge is visited.
       \end{minipage}
   \end{proof}


    

\end{enumerate}

\vspace{20pt}

\textbf{Remark:} You need to include your .pdf and .tex files in your uploaded .rar or .zip file.

%========================================================================
\end{document}
