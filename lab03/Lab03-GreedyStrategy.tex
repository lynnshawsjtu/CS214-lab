\documentclass[12pt,a4paper]{article}
\usepackage{ctex}
\usepackage{amsmath,amscd,amsbsy,amssymb,latexsym,url,bm,amsthm}
\usepackage{epsfig,graphicx,subfigure}
\usepackage{enumitem,balance}
\usepackage{wrapfig}
\usepackage{mathrsfs,euscript}
\usepackage[usenames]{xcolor}
\usepackage{hyperref}
\usepackage[vlined,ruled,linesnumbered]{algorithm2e}
\hypersetup{colorlinks=true,linkcolor=black}

\newtheorem{theorem}{Theorem}
\newtheorem{lemma}[theorem]{Lemma}
\newtheorem{proposition}[theorem]{Proposition}
\newtheorem{corollary}[theorem]{Corollary}
\newtheorem{exercise}{Exercise}
\newtheorem*{solution}{Solution}
\newtheorem{definition}{Definition}
\theoremstyle{definition}

\renewcommand{\thefootnote}{\fnsymbol{footnote}}

\newcommand{\postscript}[2]
 {\setlength{\epsfxsize}{#2\hsize}
  \centerline{\epsfbox{#1}}}

\renewcommand{\baselinestretch}{1.0}

\setlength{\oddsidemargin}{-0.365in}
\setlength{\evensidemargin}{-0.365in}
\setlength{\topmargin}{-0.3in}
\setlength{\headheight}{0in}
\setlength{\headsep}{0in}
\setlength{\textheight}{10.1in}
\setlength{\textwidth}{7in}
\makeatletter \renewenvironment{proof}[1][Proof] {\par\pushQED{\qed}\normalfont\topsep6\p@\@plus6\p@\relax\trivlist\item[\hskip\labelsep\bfseries#1\@addpunct{.}]\ignorespaces}{\popQED\endtrivlist\@endpefalse} \makeatother
\makeatletter
\renewenvironment{solution}[1][Solution] {\par\pushQED{\qed}\normalfont\topsep6\p@\@plus6\p@\relax\trivlist\item[\hskip\labelsep\bfseries#1\@addpunct{.}]\ignorespaces}{\popQED\endtrivlist\@endpefalse} \makeatother

\begin{document}
\noindent

%========================================================================
\noindent\framebox[\linewidth]{\shortstack[c]{
\Large{\textbf{Lab03-Greedy Strategy}}\vspace{1mm}\\
CS214-Algorithm and Complexity, Xiaofeng Gao, Spring 2019.}}


\begin{center}
\footnotesize{\color{red}$*$ If there is any problem, please contact TA Mingran Peng.}\par
% Please write down your name, student id and email.
\footnotesize{\color{blue}$*$ Name:\_\_\_\_\_\_\_\_\_  \quad Student ID:\_\_\_\_\_\_\_\_\_ \quad Email: \_\_\_\_\_\_\_\_\_\_\_\_}
\end{center}
\begin{enumerate}
    \item
    Suppose there is a street with length $n$, described by an array $A[1...n]$ where $A[i]=1$ means that there is a house at position $i$ and $A[i]=0$ means position $i$ is vacant.\par
	According to some law, every house must be protected by fire hydrant. If a fire hydrant is placed at position $i$, then all houses at postion $i-1,i,i+1$ will be considered protected. Note that hydrants can be placed at the same place with a house.\par
	Using what you learnt in class, please design an algorithm that computes the minimum number of hydrants needed to protect all houses. You need to write pseudo code, analyze the time complexity,  and prove its correctness.\par
%    \begin{proof}
%        Uncomment this block to write your proof.
%    \end{proof}

    \item
\begin{enumerate}
\item
    Given a set $A$ containing $n$ real numbers, and you are allowed to choose $k$ numbers from $A$. The bigger the sum of the chosen numbers is, the better. What is your algorithm to choose? Prove its correctness using \textbf{Matroid}.\par
\textbf{Remark:} This is a very easy problem. Denote $\mathbf{C}$ be the collection of all subsets of $A$ that contains no more than $k$ elements. Try to prove $(A,\mathbf{C})$ is a matroid.\par
%    \begin{proof}
%        Uncomment this block to write your proof.
%    \end{proof}
\item
Consider that $B_1,B_2 ... B_n$ are $n$ disjoint sets, and let $d_i$ be integers with $0\leq d_{i}\leq |B_{i}|$. Define $\mathbf{C}$ is a collection of set $X\subseteq \cup^{n}_{i=1} B_i$, where $X$ has such property:
$$\forall i\in \{1,2,3 ... n\},  |X\cap B_{i}|\leq d_{i}$$
Prove that $(\cup^{n}_{i=1} B_i,\mathbf{C})$ is a matroid.\par
\textbf{Remark:} You may easily find that the matroid in (a) is a special case of matroid in (b).
%    \begin{proof}
%        Uncomment this block to write your proof.
%    \end{proof}


\end{enumerate}
  

    

\end{enumerate}

\vspace{20pt}

\textbf{Remark:} You need to include your .pdf and .tex files in your uploaded .rar or .zip file.

%========================================================================
\end{document}
