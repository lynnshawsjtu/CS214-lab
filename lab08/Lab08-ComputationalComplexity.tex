\documentclass[12pt,a4paper]{article}
\usepackage{ctex}
\usepackage{amsmath,amscd,amsbsy,amssymb,latexsym,url,bm,amsthm}
\usepackage{epsfig,graphicx,subfigure}
\usepackage{enumitem,balance}
\usepackage{wrapfig}
\usepackage{mathrsfs,euscript}
\usepackage[usenames]{xcolor}
\usepackage{hyperref}
\usepackage[vlined,ruled,linesnumbered]{algorithm2e}
\usepackage{array}
\hypersetup{colorlinks=true,linkcolor=black}

\newtheorem{theorem}{Theorem}
\newtheorem{lemma}[theorem]{Lemma}
\newtheorem{proposition}[theorem]{Proposition}
\newtheorem{corollary}[theorem]{Corollary}
\newtheorem{exercise}{Exercise}
\newtheorem*{solution}{Solution}
\newtheorem{definition}{Definition}
\theoremstyle{definition}

\renewcommand{\thefootnote}{\fnsymbol{footnote}}

\newcommand{\postscript}[2]
 {\setlength{\epsfxsize}{#2\hsize}
  \centerline{\epsfbox{#1}}}

\renewcommand{\baselinestretch}{1.0}

\setlength{\oddsidemargin}{-0.365in}
\setlength{\evensidemargin}{-0.365in}
\setlength{\topmargin}{-0.3in}
\setlength{\headheight}{0in}
\setlength{\headsep}{0in}
\setlength{\textheight}{10.1in}
\setlength{\textwidth}{7in}
\makeatletter \renewenvironment{proof}[1][Proof] {\par\pushQED{\qed}\normalfont\topsep6\p@\@plus6\p@\relax\trivlist\item[\hskip\labelsep\bfseries#1\@addpunct{.}]\ignorespaces}{\popQED\endtrivlist\@endpefalse} \makeatother
\makeatletter
\renewenvironment{solution}[1][Solution] {\par\pushQED{\qed}\normalfont\topsep6\p@\@plus6\p@\relax\trivlist\item[\hskip\labelsep\bfseries#1\@addpunct{.}]\ignorespaces}{\popQED\endtrivlist\@endpefalse} \makeatother

\begin{document}
\noindent

%========================================================================
\noindent\framebox[\linewidth]{\shortstack[c]{
\Large{\textbf{Lab08-Computational Complexity}}\vspace{1mm}\\
CS214-Algorithm and Complexity, Xiaofeng Gao, Spring 2019.}}
\begin{center}
\footnotesize{\color{red}$*$ If there is any problem, please contact TA Jiahao Fan or TA Mingran Peng.}

% Please write down your name, student id and email.
\footnotesize{\color{blue}$*$ Name:Lynn Xiao  \quad Student ID:\_\_\_\_\_\_\_\_\_ \quad Email: \_\_\_\_\_\_\_\_\_\_\_\_}
\end{center}

\begin{enumerate}
    \item
    Design a one-tape TM $M$ that computes the function $f(x, y) = x - y$, where $x$ and $y$ are positive integers ($x > y$). The alphabet is $\{1, 0, \Box, \triangleright, \triangleleft\}$, and the inputs are $x$ 1's, $\Box$ and $y$ 1's. Below is the initial configuration for input $x=7$ and $y=3$. The result $z=f(x, y)$ should also be represented in the form of $z$ 1's on the tape with the pattern of $\triangleright 111 \cdots 111 \triangleleft$.

    \begin{center}
    \begin{tabular}{ll|c|c|c|c|c|c|c|c|c|c|c|c|c|c}
    	\cline{2-16}
    	Init:& & $\triangleright$ &  1  & 1 & 1 & 1 & 1 & 1 & 1 & $\Box$ & 1 & 1 & 1 & $ \triangleleft$ & \\
    	\cline{2-16}
    	\multicolumn{2}{c}{} & \multicolumn{1}{c}{$\uparrow$} & \multicolumn{11}{c}{}\\
    	\multicolumn{2}{c}{} & \multicolumn{1}{c}{$q_S$} & \multicolumn{11}{c}{}\\
    \end{tabular}
    \end{center}

    \begin{enumerate}
        \item
        Please describe your design and then write the specifications of $M$ in the form like $\langle q_S, \triangleright \rangle \rightarrow \langle q_1, \triangleright,  R\rangle$. Explain the transition functions in detail.

        \item
        Please draw the state transition diagram using Microsoft Visio.

        \item
        Show briefly and clearly the whole process from initial to final configurations for input $x = 7$ and $y = 3$.
    \end{enumerate}

    \item
    What is the ``certificate'' and ``certifier'' for the following problems?
    \begin{enumerate}
        \item
        \emph{PARTITION}: Given a finite set $A$ and a size $s(a) \in \mathbb{Z}$ for each $a \in A$, is there a subset $A' \subseteq A$ such that $\sum_{a \in A'}s(a) = \sum_{a \in A-A'}s(a)$ ?

        \item
        \emph{CLIQUE}: Given a graph $G = (V, E)$ and a positive integer $K \leq |V|$, is there a subset $V' \subseteq V$ with $|V'| \geq K$ such that every two vertices in $V'$ are joined by an edge in $E$ ?

        \item
        \emph{ZERO-ONE INTEGER PROGRAMMING}: Given an integer $m \times n$ matrix $A$ and an integer $m$-vector $b$, is there an integer $n$-vector $x$ with elements in the set $\{0, 1\}$ such that $Ax \leq b$ ?
    \end{enumerate}
	\begin{solution}
		~\\
		(a) certificate: a partition of set A.  
		
		certifier: compute the some of each part and check if they are the same.
		
		(b) certificate: a subset of $V$
		
		certifier: check if the subset contains at least $k$ elements and there is an edge between each pair of the nodes in the subset.
		
		(c) certificate: an assignment of {0,1} to n-vector x
		
		certifier: compute $Ax$ and check if each element of it is smaller than the corresponding element in $b$.
	\end{solution}
    \item
    \emph{SUBSET SUM}: Given a finite set $A$, a size $s(a) \in \mathbb{Z}$ for each $a \in A$ and an integer $B$, is there a subset $A' \subseteq A$ such that $\sum_{a \in A'}s(a) = B$?

    \emph{KNAPSACK}: Given a finite set $A$, a size $s(a) \in \mathbb{Z}$ and a value $v(a) \in \mathbb{Z}$ for each $a \in A$ and integers $B$ and $K$, is there a subset $A' \subseteq A$ such that $\sum_{a \in A'}s(a) \leq B$ and $\sum_{a \in A'}v(a) \geq K$?

    \begin{enumerate}
    \item
    Prove \emph{PARTITION} $\leq_p$ \emph{SUBSET SUM}.
	\begin{proof}
		~\\
		Instance of PARTITION: A finite set $A$ and a size $s(a)\in\mathbb{Z}$, there is a subset $A' \subseteq A$ that satisfies  $\sum_{a \in A'}s(a) = \sum_{a \in A-A'}s(a)$.
		
		Instance of SUBSETSUM: The same set $A$ with the instance of PARTITION and let B be the half of the sum of the elements in $A$.
		
		If $A'$ satisfies the requirement of the instance of PARTITION, we can easily know that it also satisfies the requirement of the instance of SUBSETSUM and vice versa.
	\end{proof}
    \item
    Prove \emph{SUBSET SUM} $\leq_p$ \emph{KNAPSACK}.
	\begin{proof}
		~\\
		Instance of PARTITION: A finite set $A$ and a size $s(a)\in\mathbb{Z}$ for each $a\in A$ and an integer B. A subset $A' \subseteq A$ such that $\sum_{a \in A'}s(a) = B$.
		
		Instance of KNAPSACK: The same set with set in the instance of PARTITION, and the same integer B. Let value of every element in A be 1 and $K=B$.
		
			If $A'$ satisfies the requirement of the instance of SUBSETSUM, we can easily know that it also satisfies the requirement of the instance of KNAPSACK and vice versa.
	\end{proof}
    \end{enumerate}

    \item
    \emph{3-SAT}: Given a set $U$ of variables, a collection $C$ of clauses over $U$ such that each clause $c \in  C$ has $|c| = 3$, is there a satisfying truth assignment for $C$?

    Prove \emph{3-SAT} $\leq_p$ \emph{CLIQUE}.
	\begin{proof}
		~\\
		Given an instance of 3-SAT, we construct an instance$(G,k)$ that has a subset equal or larger than $k$.
		
		G contains 6 vertices for each clause, one for each literal. Initially, each pair of vertices is connected. Then we only remain on vertex between different clause. Also, the connection between $x_k$ and $\bar x_k$
		in the same clause.
		
		If we can find a subset of size $k$ in $G$ satisfies the requirement of CLIQUE, the we can assign true to each element in the subset so that $\phi$ is true which satisfies 3-SAT. Also we can prove this statement vice versa.
		
	
	\end{proof}

\item Algorithm class is a democratic class. Denote class as a finite set $S$ containing every students. Now students decided to raise a student union $S' \subseteq S$ with $|S'|\leq K$ .\par
As for the members of the union, there are many different opinions. An opinion is a set $S_o\subseteq S$. Note that number of opinions has nothing to do with number of students.\par
The question is whether there exists such student union $S' \subseteq S$ with $|S'|\leq K$, that $S'$ contains at least one element from each opinion. We call this problem \emph{ELECTION} problem, prove that it is NP-complete.

\begin{proof}
~\\
Given all opinions and a student union $S' \subseteq S$, we can check if $S'$ contains at least one element from each opinion one by one. So we can certificate it in polynomial time and prove it's a NP problem.

From the course we know that $3-SAT$ is a NP-complete problem and $3-SAT\le_pVERTEX-COVER$. Therefore $VERTEX-COVER$ is a NP-complete problem.

We claim that \emph{VERTEX-COVER} problem can reduce to \emph{ELECTION} problem. Given an arbitrary graph, let each vertex be a student and each edge be an opinion. Two corresponding vertices of the edge is the elements of the opinion.

To check if there is a subset whose size is equal or smaller than $k$ that covers all edges, we only have to check if there is a student union whose size is equal or smaller than $k$ that contains at least one element from each opinion. Therefore, $\emph{VERTEX-COVER}\le_p\emph{ELECTION}$. So \emph{ELECTION} is np-complete.
\end{proof}
\end{enumerate}

\vspace{20pt}

\textbf{Remark:} You need to include your .pdf and .tex files in your uploaded .zip file.

%========================================================================
\end{document}
