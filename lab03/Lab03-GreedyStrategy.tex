\documentclass[12pt,a4paper]{article}
\usepackage{ctex}
\usepackage{amsmath,amscd,amsbsy,amssymb,latexsym,url,bm,amsthm}
\usepackage{epsfig,graphicx,subfigure}
\usepackage{enumitem,balance}
\usepackage{wrapfig}
\usepackage{mathrsfs,euscript}
\usepackage[usenames]{xcolor}
\usepackage{hyperref}
\usepackage[vlined,ruled,linesnumbered]{algorithm2e}
\hypersetup{colorlinks=true,linkcolor=black}

\newtheorem{theorem}{Theorem}
\newtheorem{lemma}[theorem]{Lemma}
\newtheorem{proposition}[theorem]{Proposition}
\newtheorem{corollary}[theorem]{Corollary}
\newtheorem{exercise}{Exercise}
\newtheorem*{solution}{Solution}
\newtheorem{definition}{Definition}
\theoremstyle{definition}

\renewcommand{\thefootnote}{\fnsymbol{footnote}}

\newcommand{\postscript}[2]
 {\setlength{\epsfxsize}{#2\hsize}
  \centerline{\epsfbox{#1}}}

\renewcommand{\baselinestretch}{1.0}

\setlength{\oddsidemargin}{-0.365in}
\setlength{\evensidemargin}{-0.365in}
\setlength{\topmargin}{-0.3in}
\setlength{\headheight}{0in}
\setlength{\headsep}{0in}
\setlength{\textheight}{10.1in}
\setlength{\textwidth}{7in}
\makeatletter \renewenvironment{proof}[1][Proof] {\par\pushQED{\qed}\normalfont\topsep6\p@\@plus6\p@\relax\trivlist\item[\hskip\labelsep\bfseries#1\@addpunct{.}]\ignorespaces}{\popQED\endtrivlist\@endpefalse} \makeatother
\makeatletter
\renewenvironment{solution}[1][Solution] {\par\pushQED{\qed}\normalfont\topsep6\p@\@plus6\p@\relax\trivlist\item[\hskip\labelsep\bfseries#1\@addpunct{.}]\ignorespaces}{\popQED\endtrivlist\@endpefalse} \makeatother

\begin{document}
\noindent

%========================================================================
\noindent\framebox[\linewidth]{\shortstack[c]{
\Large{\textbf{Lab03-Greedy Strategy}}\vspace{1mm}\\
CS214-Algorithm and Complexity, Xiaofeng Gao, Spring 2019.}}


\begin{center}
\footnotesize{\color{red}$*$ If there is any problem, please contact TA Mingran Peng.}\par
% Please write down your name, student id and email.
\footnotesize{\color{blue}$*$ Name:Lynn Xiao \quad Student ID:\_\_\_\_\_\_\_\_\_ \quad Email: \_\_\_\_\_\_\_\_\_\_\_\_}
\end{center}
\begin{enumerate}
    \item
    Suppose there is a street with length $n$, described by an array $A[1...n]$ where $A[i]=1$ means that there is a house at position $i$ and $A[i]=0$ means position $i$ is vacant.\par
	According to some law, every house must be protected by fire hydrant. If a fire hydrant is placed at position $i$, then all houses at postion $i-1,i,i+1$ will be considered protected. Note that hydrants can be placed at the same place with a house.\par
	Using what you learnt in class, please design an algorithm that computes the minimum number of hydrants needed to protect all houses. You need to write pseudo code, analyze the time complexity,  and prove its correctness.\par
    \begin{proof}
     ~\\
     We start at position 1 and check each position sequentially. We will place a fire hydrant at position i+1, then the position i and position i+2 will be both under protection. Then we will continue the procedure at the position i+3. The algorithm will shown in Alg.1.
     
     \begin{minipage}[t]{0.8\textwidth}
     	\begin{algorithm}[H]
     		\KwIn{An array $A[1,\cdots,n]$}
     		\KwOut{the minimum num of hydrant}
     		\BlankLine
     		\caption{GreedyPlacing}
     		\label{Alg-selectionsort}
     		\BlankLine
     		$i\leftarrow 1$\;
     		$N\leftarrow0$\;
     		\For{$i\leftarrow 1$ \textbf{to} $n$}{
     			\If{A[i]=0}{$i\leftarrow i+1$}
     			\Else{$N\leftarrow N+1$\;
     			$i\leftarrow i+3$}
     		}
     		\Return{N}
     		
     	\end{algorithm}
     \end{minipage}
     
     It is obvious that the time complexity of this algorithm is O(n).
     
     Correctness:
     
     We can prove the correctness of the greedy algorithm by contradiction.
     
     Assume that greedy algorithm does not provide the optimal solution.
     
     Let $i_1,i_2,...,i_k(i_1\textless i_2\textless...\textless i_k)$ be the set of positions selected by greedy algorithm.
     
     Let $j_1,j_2,...,j_m(j_1\textless j_2\textless...\textless j_m)$ be the set of positions in an optimal solution with $i_1=j_1, i_2=j_2,...,i_r=j_r$ for the largest possible value of $r$. Then we know position 1 to $i_r+1$ is protected.
     
     Now consider the $(r+1)^{th}$ hydrant. Suppose the next closest to $i_r$ house without protection is at position $x$. To protect the house at position $x$, we can place the hydrant at position $x-1$, $x$ or $x+1$.
     
     In our greedy algorithm, we will always place the hydrant at position $x+1$. If the OPT choose the position $x$ or $x-1$, there will be unnecessary waste of space. So the greedy algorithm is optimal.
     
     \end{proof}

    \item
\begin{enumerate}
\item
    Given a set $A$ containing $n$ real numbers, and you are allowed to choose $k$ numbers from $A$. The bigger the sum of the chosen numbers is, the better. What is your algorithm to choose? Prove its correctness using \textbf{Matroid}.\par
\textbf{Remark:} This is a very easy problem. Denote $\mathbf{C}$ be the collection of all subsets of $A$ that contains no more than $k$ elements. Try to prove $(A,\mathbf{C})$ is a matroid.\par
    \begin{proof}
    	~\\
        Denote $\mathbf{C}$ be the collection of all subsets of $A$ that contains no more than $k$ elements. If $D\subset E$, $E\in\mathbf{C}$, obviously $|D|\textless|E|\le k$ and $D\subset A$. Thus  $(A,\mathbf{C})$ is an independent system. We apply the Greedy-MAX algorithm to $(A,\mathbf{C})$. We define the associated strictly positive function $c(x)$ to each element $x\in A$. Let $c(x)=x-min\{x|x\in A\}+1$. The greedy-MAX algorithm is shown in Alg.2.
        
        \begin{minipage}[t]{0.8\textwidth}
        	\begin{algorithm}[H]
        		\KwIn{$k$,the number of elements to choose}
        		\KwOut{A set I}
        		\BlankLine
        		\caption{Greedy-MAX}
        		\label{Alg-selectionsort}
        		\BlankLine
        		Sort all elements in A into ordering $c(x_1)\ge c(x_2)\ge...\ge c(x_n)$
        		
        		$I\leftarrow\emptyset$
        		
        		\For{$i\leftarrow 1$ to n}
        		{
        			\If{$|I\cup\{x_i\}|\le k$}{$I\leftarrow I\cup\{x_i\}$}
        		}
        		\Return $I$
        	\end{algorithm}
        \end{minipage}
    
    	Correctness:Firstly we try to prove $(A,\mathbf{C})$ is a matroid.
    	
    	(i) Hereditary:  If $D\subset \mathbf{C}$, $E\in\mathbf{C}$, obviously $|D|\textless|E|\le k$ and $D\subset A$, so $D \in \mathbf{C}$ is true.
    	
    	(ii) Exchange Property: Consider $D,E\in \mathbf{C}$ with $|D|\textless|E|$. Because $|E|\le k$, $|D|\le k-1$. Choose an element $x$ in $E$ that is not in $D$. Then $|D\cup \{x\}|\le k$ and $D\cup \{x\}$ is a subset of $A$. So $D\cup \{x\} \in \mathbf{C}$ is true.
    	
    	So $(A,\mathbf{C})$ is a matroid. We can prove the correctness of Greedy-MAX algorithm to solve this problem.
    \end{proof}
\item
Consider that $B_1,B_2 ... B_n$ are $n$ disjoint sets, and let $d_i$ be integers with $0\leq d_{i}\leq |B_{i}|$. Define $\mathbf{C}$ is a collection of set $X\subseteq \cup^{n}_{i=1} B_i$, where $X$ has such property:
$$\forall i\in \{1,2,3 ... n\},  |X\cap B_{i}|\leq d_{i}$$
Prove that $(\cup^{n}_{i=1} B_i,\mathbf{C})$ is a matroid.\par
\textbf{Remark:} You may easily find that the matroid in (a) is a special case of matroid in (b).
    \begin{proof}
        	
        (i) Hereditary:  If $D\subset \mathbf{C}$, $E\in\mathbf{C}$, $\forall i\in{1,2,...,n}$, $|D\cap B_i|\le|E\cap B_i|\le d_i$, so $D \in \mathbf{C}$ is true.
        
        (ii) Exchange Property: Consider $D,E\in \mathbf{C}$ with $|D|\textless|E|$. As $|D|\textless|E|$, at least for one set $B_x$ there exists $|D\cap B_x|\textless|E\cap B_x|\le d_x$.
        
        Choose an element $x\in B_x$ in $E$ that is not in $D$. For the set $B_x$, $|(D\cup\{x\})\cap B_x|$=$|D\cap B_x|+1\le d_x$. Since $B_1,B_2 ... B_n$ are disjoint sets, $x$ does not belong to any other set except for $B_x$. So $|(D\cup\{x\})\cap B_i|$=$|D\cap B_i|\le d_i$. Therefore we prove $D\cup \{x\} \in \mathbf{C}$.
        
        So $(\cup^{n}_{i=1} B_i,\mathbf{C})$ is a matroid.
    \end{proof}


\end{enumerate}
  

    

\end{enumerate}

\vspace{20pt}

\textbf{Remark:} You need to include your .pdf and .tex files in your uploaded .rar or .zip file.

%========================================================================
\end{document}
