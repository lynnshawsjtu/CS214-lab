\documentclass[12pt,a4paper]{article}
\usepackage{ctex}
\usepackage{amsmath,amscd,amsbsy,amssymb,latexsym,url,bm,amsthm}
\usepackage{epsfig,graphicx,subfigure}
\usepackage{enumitem,balance}
\usepackage{wrapfig}
\usepackage{mathrsfs,euscript}
\usepackage[usenames]{xcolor}
\usepackage{hyperref}
\usepackage{booktabs}
\usepackage[vlined,ruled,linesnumbered]{algorithm2e}
\hypersetup{colorlinks=true,linkcolor=black}

\newtheorem{theorem}{Theorem}
\newtheorem{lemma}[theorem]{Lemma}
\newtheorem{proposition}[theorem]{Proposition}
\newtheorem{corollary}[theorem]{Corollary}
\newtheorem{exercise}{Exercise}
\newtheorem*{solution}{Solution}
\newtheorem{definition}{Definition}
\theoremstyle{definition}

\renewcommand{\thefootnote}{\fnsymbol{footnote}}

\newcommand{\postscript}[2]
 {\setlength{\epsfxsize}{#2\hsize}
  \centerline{\epsfbox{#1}}}

\renewcommand{\baselinestretch}{1.0}

\setlength{\oddsidemargin}{-0.365in}
\setlength{\evensidemargin}{-0.365in}
\setlength{\topmargin}{-0.3in}
\setlength{\headheight}{0in}
\setlength{\headsep}{0in}
\setlength{\textheight}{10.1in}
\setlength{\textwidth}{7in}
\makeatletter \renewenvironment{proof}[1][Proof] {\par\pushQED{\qed}\normalfont\topsep6\p@\@plus6\p@\relax\trivlist\item[\hskip\labelsep\bfseries#1\@addpunct{.}]\ignorespaces}{\popQED\endtrivlist\@endpefalse} \makeatother
\makeatletter
\renewenvironment{solution}[1][Solution] {\par\pushQED{\qed}\normalfont\topsep6\p@\@plus6\p@\relax\trivlist\item[\hskip\labelsep\bfseries#1\@addpunct{.}]\ignorespaces}{\popQED\endtrivlist\@endpefalse} \makeatother

\begin{document}
\noindent

%========================================================================
\noindent\framebox[\linewidth]{\shortstack[c]{
\Large{\textbf{Lab01-Algorithm Analysis}}\vspace{1mm}\\
CS214-Algorithm and Complexity, Xiaofeng Gao, Spring 2019.}}
\begin{center}
\footnotesize{\color{red}$*$ If there is any problem, please contact TA Mingran Peng. Also please use English in homework.}

% Please write down your name, student id and email.
\footnotesize{\color{blue}$*$ Name:Lynn Xiao  \quad Student ID:\_\_\_\_\_\_\_\_\_ \quad Email: \_\_\_\_\_\_\_\_\_\_\_\_}
\end{center}

\begin{enumerate}


\item Read Algorithm \ref{Alg-selectionsort} and Algorithm \ref{Alg-cocktailsort} carefully. \par

\begin{minipage}[t]{0.45\textwidth}
\begin{algorithm}[H]
\KwIn{An array $A[1,\cdots,n]$}
\KwOut{$A[1,\cdots,n]$ sorted nonincreasingly}
\BlankLine
\caption{SelectionSort}
\label{Alg-selectionsort}
\BlankLine
	$i\leftarrow 1$\;
	\For{$i\leftarrow 1$ \textbf{to} $n-1$}{
		$max\leftarrow A[i];pos\leftarrow i$\;
		\For{$j\leftarrow i+1$ \textbf{to} $n$}{
			\If{$A[j] > max$}{
				$max\leftarrow A[j]$\;
				$pos\leftarrow j$\;
			}
		
		}
		swap $A[pos]$ and $A[i]$\;
	}

\end{algorithm}
\end{minipage}
\hfill
\begin{minipage}[t]{0.45\textwidth}
\begin{algorithm}[H]
\KwIn{An array $A[1,\cdots,n]$}
\KwOut{$A[1,\cdots,n]$ sorted nonincreasingly}
\BlankLine
\caption{CocktailSort}
\label{Alg-cocktailsort}
\BlankLine
	$i\leftarrow 1;$ $j\leftarrow n;$$sorted\leftarrow false$\;
	\While{\textbf{not} sorted}{
		$sorted \leftarrow true$\;
		\For{$k\leftarrow i$ \textbf{to} $j-1$}{
			\If{$A[k] < A[k+1]$}{
				swap $A[k]$ and $A[k+1]$\;
				$sorted\leftarrow false$\;
			}
		}
		$j\leftarrow j-1$\;
		

		\For{$k\leftarrow j$ \textbf{downto} $i+1$}{
			\If{$A[k-1] < A[k]$}{
				swap $A[k-1]$ and $A[k]$\;
				$sorted\leftarrow false$\;
			}
		}
		$i\leftarrow i+1$\;
	}
\end{algorithm}
\end{minipage}

   
Fill in the blanks and explain your answers. You need to answer when the best case and the worst case happen. {\color{blue} (Hint: if it's both $O(g)$ and $\Omega(g)$, just answer $\Theta(g)$)}
\begin{table}[!h]

\label{Tab-compare}
	\centering
	\begin{tabular}{c|c| c }
		\toprule[2pt]
		\textbf{Algorithm} & \textbf{Time Complexity} & \textbf{Space Complexity} \\
		\hline
		\hline
		$InsertionSort$& $O(n^2)$, $\Omega(n)$  &  $\Theta(1)$  \\
		
		$CocktailSort$ &$O(n^2)$, $\Omega(n)$  & $\Theta(1)$ \\

		$SelectionSort$ &$\Theta(n^2)$  & $\Theta(1)$  \\
		\bottomrule[2pt]


	\end{tabular}
\end{table}
    \begin{solution}
    	~\\
     	(i) Analysis of $InsertionSort$
     	
     	The best case: the array has been sorted non-increasingly. We only need to do $n-1$ times of comparison. So the time complexity is $\Omega(n)$.
     	
     	The worst case: the array is in an increasing order.So the times of comparison will be $\sum_{i=0}^{n-1} i=\frac{n(n-1)}{2}$.
     	
     	space complexity: the extra space only contains the space of constant number of variables, so the space complexity is $\Theta(1)$.
     	
     	(ii) Analysis of $CocktailSort$.
     	
     	We will first show how the algorithm works. Each iteration of $CocktailSort$ is broken up
     	into following two steps.
     	
     	After the first loop range from $i$ to $j-1$, the smallest number will be put at the position $j-1$. After the second loop range from $j$ down-to $i+1$, the largest number will be put at position $i$. So this algorithm works properly to sort the array non-increasingly.
     	
     	The best case: the array has been sorted non-increasingly. We only need to do $2n-3$ times of comparison. So the time complexity is $\Omega(n)$.
     	
     	The worst case: the array is in an increasing order. We can get the times of comparison is $\frac{n(n-1)}{2}$. So the time complexity is $\Omega(n)$.
     	
     	space complexity: the extra space only contains the space of constant number of variables, so the space complexity is $\Theta(1)$.
     	
     	(iii) Analysis of $SelectionSort$
     	Regardless of the order of the array, $SelectionSort$ will take constant times of comparison which is $\frac{n(n-1)}{2}$.
     	
     	Also it is obviously that the space complexity is $\Theta(1)$.
    \end{solution}

    \item
    Let us assume that you have learned two type of data structures: \textbf{Stack} and \textbf{Queue}. \textbf{Stack} has two operations: $push$ and $pop$, while \textbf{Queue} also has two operations: $enqueue$ and $dequeue$.\par
Now you have two \textbf{Stacks}, how can you use them to simulate a \textbf{Queue}?
\begin{enumerate}
\item Briefly explain your approach. (Pseudo code is not needed.)
\item Give the time complexity of $enqueue$ and $dequeue$ operations of the simulated \textbf{Queue}. Use \textbf{potential function} for amortized analysis.
\end{enumerate}
    \begin{solution}
    	~\\
       (a) 
       
       Assume the two stacks are \textbf{stack1} and \textbf{stack2}.
       
       (i) $enqueue(x)$:push $x$ to \textbf{stack1};
       
       (ii) $dequeue(x)$: If both two stacks are empty, then raise an exception.
       
       If \textbf{stack2} is empty, then pop all the numbers and push them to \textbf{stack2}. Finally pop the top of \textbf{stack2}.
       
       (b)
       
       Considering we have n operations, including $enqueue$ and $dequeue$.
       
       Let $C_{enq}$ be the cost for $enqueue$ and $C_{deq}$ be the cost for $dequeue$. Then $C_{eq}=1$ then $C_{dq}$ is dependent on whether \textbf{stack2} is empty. When \textbf{stack2} is empty, $C_{dq}$ is 1 and when \textbf{stack2} is not empty, $C_{dq}$ is 2$size$(\textbf{stack1})+1.
       
       So we can define a potential function, $\Phi(S_i)=2$$size$$(\textbf{stack1})$. It is easy to know that $\Phi(S_i)\ge\Phi(S_0)$.
       
       Then amortized cost setting $\hat{C}_{i}=C_{i}+\Phi(S_i)-\Phi(S_{i-1})$.
    	
    	After the operation of $enqueue$, the size of \textbf{Stack1} will increase 1.
    	
      So the amortized cost of $enqueue$ is:
      
      $\hat{C}_{enq}=C_{i}+\Phi(S_i)-\Phi(S_{i-1})=3$ 
      
       The amortized cost of $dequeue$ when \textbf{Stack2} is not empty is 
       
        $\hat{C}_{deq}=C_{i}+\Phi(S_i)-\Phi(S_{i-1})=1$
        
        because the size of \textbf{Stack1} will remain.
        
         The amortized cost of $dequeue$ when \textbf{Stack2} is empty is 
        
        $\hat{C}_{deq}=C_{i}+\Phi(S_i)-\Phi(S_{i-1})=2$$size$$(\textbf{Stack1})+1+0-2$$size$$(\textbf{Stack1})=1$
        
        Finally, we have $\sum_{i=1}^nC_i\le\sum_{i=1}^n\hat{C}_i=3n$.
        
        So the amortized cost of each operation can be $O(1)$.
    \end{solution}


\end{enumerate}

\vspace{20pt}

\textbf{Remark:} You need to include your .pdf and .tex files in your uploaded .rar or .zip file.

%========================================================================
\end{document}
